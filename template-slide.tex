% Apresentações em widescreen. Outros valores possíveis: 1610, 149, 54, 43 e 32.
% Por padrão, as apresentações são no formato 4:3 (sem o aspectratio).
\documentclass[aspectratio=169]{beamer}	 	

\renewcommand\textbullet{\ensuremath{\bullet}}

\usetheme{Pittsburgh}
\usecolortheme{default}
\usefonttheme[onlymath]{serif}			% para fontes matemáticas
% Enconte mais temas e cores em http://www.hartwork.org/beamer-theme-matrix/ 
% Veja também http://deic.uab.es/~iblanes/beamer_gallery/index.html
% Customizações de Cores: fg significa cor do texto e bg é cor do fundo
\setbeamercolor{normal text}{fg=black}
\setbeamercolor{alerted text}{fg=red}
\setbeamercolor{author}{fg=black}
\setbeamercolor{institute}{fg=black}
\setbeamercolor{date}{fg=black}
\setbeamercolor{frametitle}{fg=black}
\setbeamercolor{framesubtitle}{fg=brown}
\setbeamercolor{block title}{bg=black, fg=white}	%Cor do título
\setbeamercolor{block body}{bg=gray, fg=black}	%Cor do texto (bg= fundo; fg=texto)
\setbeamercolor{section in toc}{fg=black}
\setbeamercolor{section number projected}{bg=white,fg=black}
\setbeamercolor{titlelike}{fg=black}
% ---
% PACOTES
% ---
\usepackage[alf]{abntex2cite}	% Citações padrão ABNT
\usepackage[brazil]{babel}		% Idioma do documento
\usepackage{color}				% Controle das cores
\usepackage[T1]{fontenc}		% Selecao de codigos de fonte.
\usepackage{graphicx}			% Inclusão de gráficos
\usepackage[utf8]{inputenc}		% Codificacao do documento (conversão automática dos acentos)
\usepackage{txfonts}			% Fontes virtuais
\usepackage{bmpsize}
\usepackage[absolute,overlay]{textpos}
% ---

% --- Informações do documento ---
\title{Titulo}
\author{Nome}
%\institute{Universidade Federal do Pará \par Centro de Tecnologia da Informação e Comunicação}
\date{\today}
% ---

% ----------------- INÍCIO DO DOCUMENTO --------------------------------------
\begin{document}
\usebackgroundtemplate{\includegraphics[width=1.5cm,height=\paperheight]{imagens/imagem.png}}
% ----------------- NOVO SLIDE --------------------------------
\begin{frame}
\begin{minipage}{\linewidth}
  \centering
  \begin{tabular}{cc}
    \begin{tabular}{c}
      \textbf{Universidade Federal do Pará} \\ 
      \textbf{Centro de Tecnologia da Informação e Comunicação}
    \end{tabular}
    &
    \begin{tabular}{c}
     \includegraphics[width=4cm]{imagens/ctic-logo.png}
    \end{tabular}
  \end{tabular}
\end{minipage}
\titlepage
\end{frame}
% ----------------- NOVO SLIDE --------------------------------
\begin{frame}{Sumário}
  \tableofcontents
\end{frame}
% ----------------- NOVO SLIDE --------------------------------
\section{Seção}
\subsection{Subseção}
\begin{frame}{Titulo do Slide}
  \framesubtitle{Sub titulo}
\end{frame}
% ----------------- NOVO SLIDE --------------------------------
\end{document}
